\documentclass[aps,twocolumn,secnumarabic,balancelastpage,amsmath,amssymb,nofootinbib]{revtex4}

\usepackage{color}         % produces boxes or entire pages with colored backgrounds
\usepackage{graphics}      % standard graphics specifications
\usepackage{graphicx}
\usepackage[colorlinks=true]{hyperref}
                                        
%%%%%%%%%%%%%%%%%%%%%%%%%%%%%%%%%%%%%%%%%%%%%%%%%%%%%%%%%%%%%%%%%%%%%%%%%%%%%%%%%%%%%%%%%%%%%%%%%%%%%%%%%%%%%%%%%%%%%%%%%%%%%%%%%%%%%%%%%%%%%%%%%%%%%%%%%%%%%%%%%%%%%%%%%%%%%%%%%%%%%%%%%%%%%%%%%%%%%%%%
                                        
\begin{document}
\title{A Study of the Energy Quantization in Atomic Structure}
\author{Joseph Hartsell}
\email{jhartsell3@gatech.edu}
\author{M. Lane}
\email{mlane3@gatech.edu}
\date{\today}
\affiliation{Georgia Institute of Technology, School of Physics}

\begin{abstract}
	randome abstract
\end{abstract}

\maketitle

\section{Introduction}

Physics of the early twentieth centry is characterized almost entirely by quantum effects. This began with Max Planck's solution to the Ultraviolet Catastrophe in 1900 followed by Einstein's explanation of the photoelectric effect. Together these two experiments brought quantization to light. In 1909 Rutherford introduced a new model of the atom and in 1913 Neils Bohr expanded on the model and introduced quantized energy levels to the atom.
In 1914 James Franck and Gustav Hertz developed an elegantly simple experiment to verify the Bohr model and its proposed energy levels.

\section{Theory}

Theory

\subsection{Classic Experiment Using Hg Vapor}

subsection

\end{document}
