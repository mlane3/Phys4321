\documentclass[aps,twocolumn,secnumarabic,balancelastpage,amsmath,amssymb,nofootinbib]{revtex4}

\usepackage{color}         % produces boxes or entire pages with colored backgrounds
\usepackage{graphics}      % standard graphics specifications
\usepackage{graphicx}
\usepackage[colorlinks=true]{hyperref}
                                        
%%%%%%%%%%%%%%%%%%%%%%%%%%%%%%%%%%%%%%%%%%%%%%%%%%%%%%%%%%%%%%%%%%%%%%%%%%%%%%%%%%%%%%%%%%%%%%%%%%%%%%%%%%%%%%%%%%%%%%%%%%%%%%%%%%%%%%%%%%%%%%%%%%%%%%%%%%%%%%%%%%%%%%%%%%%%%%%%%%%%%%%%%%%%%%%%%%%%%%%%
                                        
\begin{document}
\title{A Study of the Energy Quantization in Atomic Structure}
\author{J. M. Yurchesyn}
\email{jmyurchesyn1@gatech.edu}
\author{J. W. Hue}
\email{jhue3@gatech.edu}
\date{\today}
\affiliation{Georgia Institute of Technology, School of Physics}

\begin{abstract}
 as well as performing a spectroscopic analysis of the line spectra of sodium and helium.  Results of this experiment will provide confirmation of the assumptions put forth by quantum theory
 regarding the existence of discrete energy levels in atoms and will lay the experimental foundations for the detailed investigation of the atomic structure of the elements. 
\end{abstract}

\maketitle
\section{Introduction}

Prior to the experiments performed by German physicists James Franck and Gustav Hertz in the early 1900s, experimental evidence for the assumptions of quantum theory only involved the 
quantization of electromagnetic radiation, or photons.  Spectroscopic experiments up to this point were typically performed by spectral analysis of the electrical exication of gaseous vapor 
in a high-voltage vacuum tube.  The Bohr model provided an explanation of such exictations but only for certain elements.

In 1914, Franck and Hertz designed an experiment that would study the ionization of atoms through the bombardment of gaeous vapor with electrons.  Of key interest was to examine the possibility
of transferring part of an electron's kinetic energy to an atom, or more precisely, to the electrons bound to the nucleus of an atom.  The successfull observation of such measurements were a crucial
step in the development of modern physics given that it provided a new technique for the study of atomic structure.

\section{Theory}

The typical arrangement of the Franck-Hertz experiment can be described as follows (a more detailed explanation of the particular design used in our experiment can be found below): thermionic 
emission of electrons from a hot cathode are accelerated by an applied potential $V_{A}$ (typically between 0 and 45 V) within a vacuum tube containing gaeous vapors of the element of interest.  

The acclerating voltage allows us to determine the maximum kinetic energy of the emitted electrons, given by \emph{e$V_{A}$}.  The electrons then pass through a wire mesh and are subjected 
to a retarding potential $V_{R}$ established between the grid and collector (anode).  This allows the selection of electrons with a specific minimum kinetic energy, defined by \emph{e$V_{R}$}, 
since only those electrons with approximately \emph{e$V_{R}$} of energy will reach the collector and register a current.  Experimental analysis then examines the flucations in collector 
current as a function of accelerating voltage.  

\subsection{Classic Experiment Using Hg Vapor}

Inital experiments performed by Franck and Hertz used Hg vapor.  Franck and Hertz observed that when the accelerating voltage was less then 5 eV, the electrons experienced no energy loss. This suggests 
that the electrons are not energetic enough to cause exication of the Hg atoms.  As the accelerating potential is increased, more electrons are able to pass through the retarding potential causing a
steady rise in current.  

But Franck and Hertz observed that the current does not increase ad infinium.  Instead, a sharp drop in collector current was observed whenever the accelerating voltage was increased above 5 eV.  
Simliar observations were made as the accelerating voltage was increased past 5 eV.  A steady increase in collector current was again observed until the voltage reached apporximately 10 eV, at 
which point a sharp drop off in current was observed.  This pattern was observed to be repeated throughout the data: current peaks in the collector current were consistently observed in $\sim$5 eV 
intervals.  Intially, Franck and Hertz interpreted this to be the evidence of the ionization of the Hg atom, i.e. the removal of an electron from one of the valance shells by energy transfer with
the incoming electron.

The theorietical mechanism behind the effect observed by Franck and Hertz can be explained by appealing to the Bohr model of the atom first introduced by Niels Bohr in 1915.  The Bohr model assumes
that the atom can be modeled as a sort of quasi-planetary system.  The electrons are viewed as orbiting the massive positively charged nucleus where it is assumed that the radius of the electron's
orbit is several orders of magnitude larger than the radius of the nucleus.  Given this, the Bohr model can be summarized by the following four general assumptions:

\end{document}
