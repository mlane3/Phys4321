\chapter{Muon Physics}
{\emph{originally by T.\ E.\ Coan and J.\ Ye,\ \ edited by S.\ Penn}}
% \pagenumbering{arabic}


\section{Introduction}

The muon is one of nature�s fundamental �building blocks of matter�
and acts in many ways as if it were an unstable heavy electron, for
reasons no one fully understands. Discovered in 1937 by C.W. Anderson
and S.H. Neddermeyer when they exposed a cloud chamber to cosmic rays,
its finite lifetime was first demonstrated in 1941 by F. Rasetti. The
instrument described in this manual permits you to measure the charge
averaged mean muon lifetime in plastic scintillator, to measure the
relative flux of muons as a function of height above sea-level and to
demonstrate the time dilation effect of special relativity. The
instrument also provides a source of genuinely random numbers that can
be used for experimental tests of standard probability distributions.

\section{The Muon Source}

The top of earth's atmosphere is bombarded by a flux of high energy
charged particles produced in other parts of the universe by
mechanisms that are not yet fully understood. The composition of these
"primary cosmic rays" is somewhat energy dependent but a useful
approximation is that 98\% of these particles are protons or heavier
nuclei and 2\% are electrons. Of the protons and nuclei, about 87\%
are protons, 12\% helium nuclei and the balance are still heavier
nuclei that are the end products of stellar nucleosynthesis. See
Simpson in the reference section for more details.
 
The primary cosmic rays collide with the nuclei of air molecules and
produce a shower of particles that include protons, neutrons, pions
(both charged and neutral), kaons, photons, electrons and positrons.
These secondary particles then undergo electromagnetic and nuclear
interactions to produce yet additional particles in a cascade process.
Figure \ref{FigShower} indicates the general idea. Of particular
interest is the fate of the charged pions produced in the cascade.
Some of these will interact via the strong force with air molecule
nuclei but others will spontaneously decay (indicated by the arrow)
via the weak force into a muon plus a neutrino or antineutrino:

\begin{eqnarray*}
   \pi^{+} & \rightarrow & \mu^{+}\nu_{mu}  \\
   \pi^{-} & \rightarrow & \mu^{-}\bar{\nu}_{\mu}
\end{eqnarray*}

The muon does not interact with matter via the strong force but only
through the weak and electromagnetic forces. It travels a relatively
long instance while losing its kinetic energy and decays by the weak
force into an electron plus a neutrino and antineutrino. We will
detect the decays of some of the muons produced in the cascade. (Our
detection efficiency for the neutrinos and antineutrinos is utterly
negligible.)


\begin{figure}[tbp]
   \centering
   \includegraphics[height=4in]{MuonLifetime/shower}
   \caption{ Cosmic ray cascade induced by a cosmic ray proton striking
   an air molecule nucleus.}
   \label{FigShower}
\end{figure}

Not all of the particles produced in the cascade in the upper
atmosphere survive down to sea-level due to their interaction with
atmospheric nuclei and their own spontaneous decay. The flux of
sea-level muons is approximately 1 per minute per cm2 (see
http://pdg.lbl.gov for more precise numbers) with a mean kinetic
energy of about 4 GeV.
 
Careful study [http://pdg.lbl.gov] shows that the mean production
height in the atmosphere of the muons detected at sea-level is
approximately 15 km. Travelling at the speed of light, the transit
time from production point to sea-level is then 50 $\mu$sec. Since the
lifetime of at-rest muons is more than a factor of 20 smaller, the
appearance of an appreciable sea- level muon flux is qualitative
evidence for the time dilation effect of special relativity.

\section{Muon Decay Time Distribution}

The decay times for muons are easily described mathematically. Suppose
at some time $t$ we have $N(t)$ muons. If the probability that a muon
decays in some small time interval $dt$ is $\lambda \,dt$, where
$\lambda$ is a constant \emph{decay rate} that characterizes how
rapidly a muon decays, then the change $dN$ in our population of muons
is just $dN = -N(t) \,\lambda \,dt$, or $dN/N(t) = -\lambda \,dt$.
Integrating, we have $N(t) = N_{0} \,e^{-\lambda t}$, where $N(t)$ is
the number of surviving muons at some time $t$ and $N0$ is the number
of muons at $t = 0$. The \emph{lifetime} $\tau$ of a muon is the
reciprocal of $\lambda$, $\tau = 1/\lambda$. This simple exponential
relation is typical of radioactive decay.
 
Now, we do not have a single clump of muons whose surviving number we
can easily measure. Instead, we detect muon decays from muons that
enter our detector at essentially random times, typically one at a
time. It is still the case that their decay time distribution has a
simple exponential form of the type described above. By decay time
distribution $D(t)$, we mean that the time-dependent probability that
a muon decays in the time interval between $t$ and $t + \,dt$ is given
by $D(t) \,dt$. If we had started with $N_{0}$ muons, then the fraction
$-dN/N_{0}$ that would on average decay in the time interval between
$t$ and $t + \,dt$ is just given by differentiating the above relation:

\begin{eqnarray*}
   -dN & = & N_{0} \,\lambda \,e^{\lambda t}\,dt  \\
   -dN/N_{0} & = & \lambda\,e^{-\lambda t}\,dt
\end{eqnarray*}

The left-hand side of the last equation is nothing more than the decay
probability we seek, so $D(t) = \lambda \, e^{-\lambda t}$. This is
true regardless of the starting value of $N_{0}$. That is, the
distribution of decay times, for new muons entering our detector, is
also exponential with the very same exponent used to describe the
surviving population of muons. Again, what we call the muon lifetime
is $\tau = 1/\lambda$.
 
Because the muon decay time is exponentially distributed, it does not
matter that the muons whose decays we detect are not born in the
detector but somewhere above us in the atmosphere. An exponential
function always \emph{looks the same} in the sense that whether you
examine it at early times or late times, its e-folding time (time to
decrease by a factor $e$) is the same.


\section{Detector Physics}

The active volume of the detector is a plastic scintillator in the
shape of a right circular cylinder of 15 cm diameter and 12.5 cm
height placed at the bottom of the black anodized aluminum alloy tube.
Plastic scintillator is transparent organic material made by mixing
together one or more fluors with a solid plastic solvent that has an
aromatic ring structure. A charged particle passing through the
scintillator will lose some of its kinetic energy by ionization and
atomic excitation of the solvent molecules. Some of this deposited
energy is then transferred to the fluor molecules whose electrons are
then promoted to excited states. Upon radiative de-excitation, light
in the blue and near-UV portion of the electromagnetic spectrum is
emitted with a typical decay time of a few nanoseconds. A typical
photon yield for a plastic scintillator is 1 optical photon emitted
per 100 eV of deposited energy. The properties of the
polyvinyltoluene-based scintillator used in the muon lifetime
instrument are summarized in table \ref{TblScintProps}.
 
To measure the muon's lifetime, we are interested in only those muons
that enter, slow, stop and then decay inside the plastic scintillator.
Figure \ref{FigSchematic} summarizes this process. Such muons have a
total energy of only about 160 MeV as they enter the tube. As a muon
slows to a stop, the excited scintillator emits light that is detected
by a photomultiplier tube (PMT), eventually producing a logic signal
that triggers a timing clock. (See the electronics section below for
more detail.) A stopped muon, after a bit, decays into an electron, a
neutrino and an anti-neutrino. (See the next section for an important
qualification of this statement.) Since the electron mass is so much
smaller that the muon mass, $m_{\mu}/m_{e} \sim 210$, the electron
tends to be very energetic and to produce scintillator light
essentially all along its pathlength. The neutrino and anti-neutrino
also share some of the muon's total energy but they entirely escape
detection. This second burst of scintillator light is also seen by the
PMT and used to trigger the timing clock. The distribution of time
intervals between successive clock triggers for a set of muon decays
is the physically interesting quantity used to measure the muon
lifetime.

\begin{figure}[tbp]
   \centering
   \includegraphics[height=3in]{MuonLifetime/ScintillatorSchematic}
   \caption{Schematic showing the generation of the two light pulses
(short arrows) used in determining the muon lifetime. One light pulse
is from the slowing muon (dotted line) and the other is from its decay
into an electron or positron (wavey line).}
   \label{FigSchematic}
\end{figure}

\begin{table}[tbp]
   \centering
   \begin{tabular}{|c|c|}
      \hline
      Mass Density &  1.032 g/cm3 \\
      \hline
      Refractive index & 1.58  \\
      \hline
      Base material &  Polyvinyltoluene \\
      \hline
      Rise time &   0.9 ns \\
      \hline
      Fall time &  2.4 ns   \\
      \hline
      $\lambda_{\mathrm{max}}$ & 423 nm  \\
      \hline
   \end{tabular}
   \caption{ General Scintillator Properties} 
   \label{TblScintProps}
\end{table}

\section{Interaction of $\mu^{-}$'s with Matter}

The muons whose lifetime we measure necessarily interact with matter.
Negative muons that stop in the scintillator can bind to the
scintillator's carbon and hydrogen nuclei in much the same way as
electrons do. Since the muon is not an electron, the Pauli exclusion
principle does not prevent it from occupying an atomic orbital already
filled with electrons. Such bound negative muons can then interact
with protons
\[
   \mathrm{\mu^{-} + p \rightarrow n + \nu_{\mu}}
\]
 
before they spontaneously decay. Since there are now two ways for a
negative muon to disappear, the effective lifetime of negative muons
in matter is somewhat less than the lifetime of positively charged
muons, which do not have this second interaction mechanism.
Experimental evidence for this effect is shown in figure
\ref{FigDisintegration} where \emph{disintegration} curves for
positive and negative muons in aluminum are shown. (See Rossi, 1952)
The abscissa is the time interval $t$ between the arrival of a muon in
the aluminum target and its decay. The ordinate, plotted
logarithmically, is the number of muons greater than the corresponding
abscissa. These curves have the same meaning as curves representing
the survival population of radioactive substances. The slope of the
curve is a measure of the effective lifetime of the decaying
substance. The muon lifetime we measure with this instrument is an
average over both charge species so the mean lifetime of the detected
muons will be somewhat less than the free space value $\tau = 2.19703
\pm 0.00004 \mu\mathrm{sec}$.
 
\begin{figure}[tbp]
   \centering
   \includegraphics[height=5in]{MuonLifetime/Disintegration}
   \caption{Disintegration curves for positive and negative muons in aluminum. The 
ordinates at $t = 0$ can be used to determine the relative numbers of negative and positive 
muons that have undergone spontaneous decay. The slopes can be used to determine the 
decay time of each charge species. (From Rossi, p168.) }
   \label{FigDisintegration}
\end{figure}


The probability for nuclear absorption of a stopped negative muon by
one of the scintillator nuclei is proportional to $Z^{4}$, where $Z$ is the
atomic number of the nucleus [Rossi, 1952]. A stopped muon captured in
an atomic orbital will make transitions down to the K-shell on a time
scale short compared to its time for spontaneous decay [Wheeler]. Its
Bohr radius is roughly 200 times smaller than that for an electron due
to its much larger mass, increasing its probability for being found in
the nucleus. From our knowledge of hydrogenic wavefunctions, the
probability density for the bound muon to be found inside the nucleus
is proportional to $Z^{3}$. Once inside the nucleus, a muon�s probability
for encountering a proton is proportional to the number of protons
there and so scales like $Z$. The net effect is for the overall
absorption probability to scale like $Z^{4}$. Again, this effect is
relevant only for negatively charged muons.

\section{$\mu^{+}/\mu^{-}$ Charge Ratio at Ground Level}

Our measurement of the muon lifetime in plastic scintillator is an
average over both negatively and positively charged muons. We have
already seen that $\mu^{-}$�s have a lifetime somewhat smaller than
positively charged muons because of weak interactions between negative
muons and protons in the scintillator nuclei. This interaction
probability is proportional to $Z^{4}$, where $Z$ is the atomic number
of the nuclei, so the lifetime of negative muons in scintillator and
carbon should be very nearly equal. This latter lifetime
$\tau_{\mathrm{C}}$ is measured to be $\tau_{\mathrm{C}} = 2.043 �
0.003 \mu\mathrm{sec}$. [Reiter, 1960]
 
It is easy to determine the expected average lifetime
$\tau_{\mathrm{obs}}$ of positive and negative muons in plastic
scintillator. Let $\lambda^{-}$ be the decay rate per negative muon in plastic
scintillator and let $\lambda^{+}$ be the corresponding quantity for positively
charged muons. If we then let $N^{-}$ and $N^{+}$ represent the number of
negative and positive muons incident on the scintillator per unit
time, respectively, the observed average decay rate $\lambda_{\mathrm{obs}}$ 
% $\langle \lambda \rangle$ 
is given by

\begin{eqnarray*}
   \lambda_{\mathrm{obs}} & = & 
   \frac{N^{+}\lambda^{+}+N^{-}\lambda^{-}}{N^{+}+N^{-}}  \\
   & = & \frac{\rho \lambda^{+} + \lambda^{-}}{1+\rho}
\end{eqnarray*}

where $\rho \equiv N^{+}/N^{-}$ is the ratio of positive muons to
negative muons. We can define the lifetime of negative muons on the
scintillator as $\tau^{-} \equiv 1/\lambda^{-}$ and the lifetime of
positive muons as $\tau^{+} \equiv 1/\lambda^{+}$. Then the observed
average lifetime $\tau_{\mathrm{obs}}$ is given by

\begin{eqnarray*}
   \tau_{\mathrm{obs}} & = & \lambda_{\mathrm{obs}}^{-1}  \\
    & = & \frac{1+\rho}{\rho \lambda^{+} + \lambda^{-}}  \\
    & = & (1+\rho) (\frac{1}{\tau^{-}} + \frac{\rho}{\tau^{+}})^{-1}  \\
    & = & (1+\rho) \frac{\tau^{-} \, \tau^{+}}{\tau^{+} + \rho \tau^{-}}
\end{eqnarray*}

The lifetime of negative muons in plastic scintillator is
approximately the same as the lifetime in carbon,
$\tau^{-}=\tau_{\mathrm{C}}$. On the other hand, positive muons are
not captured by the scintillator nuclei. Therefore the lifetime of
positive muons, $\tau^{+}$, is equal to the lifetime of muons in free
space, $\tau_{\mu}$. Setting $\rho=1$ allows us to estimate the
average muon lifetime we expect to observe in the scintillator.

We can \emph{measure} $\rho$ for the momentum range of muons that stop
in the scintillator by rearranging the above equation:
\[
\rho = -\frac{\tau^{+}}{\tau^{-}} (\frac{\tau^{-} - 
\tau_{\mathrm{obs}}}{\tau^{+} - \tau_{\mathrm{obs}}})
\]

\section{Backgrounds}

The detector responds to any particle that produces enough
scintillation light to trigger its readout electronics. These
particles can be either charged, like electrons or muons, or neutral,
like photons, that produce charged particles when they interact inside
the scintillator. Now, the detector has no knowledge of whether a
penetrating particle stops or not inside the scintillator and so has
no way of distinguishing between light produced by muons that stop and
decay inside the detector, from light produced by a pair of
through-going muons that occur one right after the other. This
important source of background events can be dealt with in two ways.
First, we can restrict the time interval during which we look for the
two successive flashes of scintillator light characteristic of muon
decay events. Secondly, we can estimate the background level by
looking at large times in the decay time histogram where we expect few
events from genuine muon decay.

\section{Fermi Coupling Constant $\mathrm{G_F}$}

Muons decay via the weak force and the Fermi coupling constant
$\mathrm{G_F}$ is a measure of the strength of the weak force. To a
good approximation, the relationship between the muon lifetime $\tau$
and $\mathrm{G_F}$ is particularly simple:
\[
   \tau = \frac{192\,\pi^{3}\,\hbar^{7}}{ 
   \mathrm{G_F}^{2}\,m^{5}\,c^{4}}
\]
where $m$ is the mass of the muon and the other symbols have their
standard meanings. Measuring $\tau$ with this instrument and then
taking $m$ from, say, the Particle Data Group (http://www.pdg.lbl.gov)
produces a value for $\mathrm{G_F}$.

\section{Time Dilation Effect}

A measurement of the muon stopping rate at two different altitudes can
be used to demonstrate the time dilation effect of special relativity.
Although the detector configuration is not optimal for demonstrating
time dilation, a useful measurement can still be preformed without
additional scintillators or lead absorbers. Due to the finite size of
the detector, only muons with a typical total energy of about 160 MeV
will stop inside the plastic scintillator. The stopping rate is
measured from the total number of observed muon decays recorded by the
instrument in some time interval. This rate in turn is proportional to
the flux of muons with total energy of about 160 MeV and this flux
decreases with diminishing altitude as the muons descend and decay in
the atmosphere. After measuring the muon stopping rate at one
altitude, predictions for the stopping rate at another altitude can be
made with and without accounting for the time dilation effect of
special relativity. A second measurement at the new altitude
distinguishes between competing predictions.
 
A comparison of the muon stopping rate at two different altitudes
should account for the muon�s energy loss as it descends into the
atmosphere, variations with energy in the shape of the muon energy
spectrum, and the varying zenith angles of the muons that stop in the
detector. Since the detector stops only low energy muons, the stopped
muons detected by the low altitude detector will, at the elevation of
the higher altitude detector, necessarily have greater energy. This
energy difference �E(h) will clearly depend on the pathlength between
the two detector positions.
 
Vertically travelling muons at the position of the higher altitude
detector that are ultimately detected by the lower detector have an
energy larger than those stopped and detected by the upper detector by
an amount equal to �E(h). If the shape of the muon energy spectrum
changes significantly with energy, then the relative muon stopping
rates at the two different altitudes will reflect this difference in
spectrum shape at the two different energies. (This is easy to see if
you suppose muons do not decay at all.) This variation in the spectrum
shape can be corrected for by calibrating the detector in a manner
described below.
 
Like all charged particles, a muon loses energy through coulombic
interactions with the matter it traverses. The average energy loss
rate in matter for singly charged particles traveling close to the
speed of light is approximately 2 MeV/g/cm$^{2}$, where we measure the
thickness $s$ of the matter in units of g/cm$^{2}$. Here, $s = 
\rho\,x$, where $\rho$ is
the mass density of the material through which the particle is
passing, measured in g/cm3, and  $x$ is the particle�s pathlength,
measured in cm. (This way of measuring material thickness in units of
g/cm$^{2}$ allows us to compare effective thicknesses of two materials that
might have very different mass densities.) A more accurate value for
energy loss can be determined from the Bethe-Bloch equation.

\begin{eqnarray*}
   \frac{dE}{dx} & = & 
   -\frac{4\pi\,N_{\mathrm{e}}^{4}}{mc^{2}\,\beta^{2}} z^{2}\, 
   \left(\ln\frac{2\,mc^{2}\beta^{2}\gamma^{2}}{I} - \beta^{2}\right) \\
    & = & 0.3071\,\left(\mathrm{\frac{MeV}{g/cm^{2}}}\right) \, \frac{Z}{A}\, 
    \rho\, \frac{1}{\beta^{2}}\, z^{2}\, 
    \left(\ln\frac{2\,mc^{2}\beta^{2}\gamma^{2}}{I} - \beta^{2}\right)
\end{eqnarray*}

Here $N$ is the number of electrons in the stopping medium per
cm$^{3}$, $e$ is the electronic charge, $z$ is the atomic number of
the projectile, $Z$ and $A$ are the atomic number and weight,
respectively, of the stopping medium. The projectile has a velocity,
$\beta$, expressed in units of the speed of light, $c$. Its
corresponding Lorentz factor is $\gamma$. The mean excitation energy
of the stopping medium atoms, $I$, can be approximates as $I \approx
AZ$, where $A \cong 13$ eV. More accurate values for $I$, as well as
corrections to the Bethe-Bloch equation, can be found in [Leo, p26].
 
A simple estimate of the energy lost, $\Delta E$, by a muon as it travels a
vertical distance $H$ is 
\[
\Delta E = 2 H \rho_{\mathrm{air}} \mathrm{\frac{MeV}{gm/cm^{2}}} 
\]
where $\rho_{\mathrm{air}}$ is the density of air, possibly averaged
over $H$ using the density of air according to the \emph{standard
atmosphere}. The atmosphere is assumed to be isothermal and the
reduced air pressure $p = P/g$ at some height $h$ above sea level is
given by $p = p_{0} e^{-h/h_{0}}$, where $p_{0} = 1030\,
\mathrm{g/cm^{2}}$ is the total thickness of the atmosphere and $h_{0}
= 8.4$ km. The units of reduced pressure are g/cm$^{2}$, which you can
see if you recall from hydrostatics that the pressure at the base of a
stationary fluid is $P = \rho g h$. Thus the reduced pressure, $p
\equiv P/g = \rho h$ will have units of $\mathrm{g/cm^{2}}$. Then the
air density $\rho$, in familiar units of g/cm$^{3}$, can be expressed 
as $\rho = -dp/dh$.
 
In the lab frame, we let $t$ denote the transit time for a particle to
travel vertically from some height $H$ down to sea level.  In the 
rest frame of the particle, the corresponding time, $t^{\prime}$, is given by
\[
t^{\prime}=\int^{0}_{H}\frac{dh}{c\,\beta(h)\gamma(h)}
\]
Here $\beta$ and $\gamma$ have their usual relativistic meanings for
the projectile and are measured in the lab frame. Since relativistic
muons lose energy at essentially a constant rate when travelling
through a medium of mass density $\rho$, $dE/ds = C_{0}$, so we have
$dE = \rho C_{0} dh$, with $C_{0} = 2$ MeV/(g/cm$^{2}$). Also, from
the Einstein relation, $E = \gamma mc^{2}$, $dE = mc^{2} d\gamma$, so
$dh = (mc^{2}/\rho C_{0}) d\gamma$. Hence,


\begin{eqnarray*}
   t^{\prime} & = & \frac{mc}{\rho C_{0}} 
   \int^{\gamma_{2}}_{\gamma_{1}}\frac{d\gamma}{\beta \gamma}  \\
    & = & \frac{mc}{\rho C_{0}} 
    \int^{\gamma_{2}}_{\gamma_{1}}\frac{d\gamma}{\sqrt{\gamma^{2}-1}}
\end{eqnarray*}

Here $\gamma_{1}$ is the muon�s gamma factor at height $H$ and
$\gamma_{2}$ is its gamma factor just before it enters the
scintillator. We can take $\gamma_{2} = 1.5$ since we want muons that
stop in the scintillator and assume that on average stopped muons
travel halfway into the scintillator, corresponding to a distance $s =
10\, \mathrm{g/cm^{2}}$. The entrance muon momentum is then taken from
range-momentum graphs at the Particle Data Group WWW site and the
corresponding $\gamma_{2}$ computed. The lower limit of integration is
given by $\gamma_{1} = E_{1}/mc^{2}$, where $E_{1} = E_{2} + \Delta
E$, with $E_{2} = 160$ MeV. The integral can be evaluated numerically.
(See, for example,
http://people.hofstra.edu/faculty/Stefan\_Waner/RealWorld/integral/integral.html)
 
\begin{figure}[tbp]
   \centering
   \includegraphics[height=4in]{MuonLifetime/MuonMomentum}
   \caption{Muon momentum spectrum at sea level. The curves are fits 
   to various data sets. Figure from Greider, p399.}
%    (shown as geometric shapes). Figure is taken from reference [Greider, p399].
   \label{FigMuonMomentum}
\end{figure}

Hence, the ratio $R$ of muon stopping rates for the same detector at
two different positions separated by a vertical distance $H$, and
ignoring for the moment any variations in the shape of the energy
spectrum of muons, is just $R = e^{-t^{\prime}/\tau}$, where $\tau$ is the muon
proper lifetime.
 
When comparing the muon stopping rates for the detector at two
different elevations, we must remember that muons that stop in the
lower detector have, at the position of the upper detector, a larger
energy. If, say, the relative muon abundance grows dramatically with
energy, then we would expect a relatively large stopping rate at the
lower detector simply because the starting flux at the position of the
upper detector was so large, and not because of any relativistic
effects. Indeed, the muon momentum spectrum does peak, at around $p =
500$ MeV/c or so, although the precise shape is not known with high
accuracy. See figure \ref{FigMuonMomentum}.


We therefore need a way to correct for variations in the shape of the
muon energy spectrum in the region from about 160 -- 800 MeV.
(Corresponding to momentum range $p = 120$ -- 790 MeV/c.) We do this
by first measuring the muon stopping rate at two different elevations
($\Delta h = 3008$ meters between Taos, NM and Dallas, TX) and then
computing the ratio $R_{\mathrm{raw}}$ of raw stopping rates.
($R_{\mathrm{raw}} = \mathrm{Dallas/Taos} = 0.41 \pm 0.05$) Next,
using the above expression for the transit time between the two
elevations, we compute the transit time in the muon�s rest frame
($t^{\prime} = 1.32\tau$) for vertically travelling muons and
calculate the corresponding theoretical stopping rate ratio $R =
e^{-t^{\prime}/\tau} = 0.267$. We then compute the double ratio $R_{0}
= R_{\mathrm{raw}}/R = 1.5 \pm 0.2$ of the measured stopping rate
ratio to this theoretical rate ratio and interpret this as a
correction factor to account for the increase in muon flux between
about E =160 MeV and E = 600 MeV. This correction is to be used in all
subsequent measurements for any pair of elevations.
 
To verify that the correction scheme works, we take a new stopping
rate measurement at a different elevation ($h = 2133$ meters a.s.l. at
Los Alamos, NM), and compare a new stopping rate ratio measurement
with our new, corrected theoretical prediction for the stopping rate
ratio $R_{\mu} = R_{0} R = 1.6 e^{-t^{\prime}/\tau}$. We find
$t^{\prime} = 1.06 \tau$ and $R_{\mu} = 0.52 \pm 0.06$. The raw
measurements yield $R_{\mathrm{raw}} = 0.56 \pm 0.01$, showing good agreement.
 
For your own time dilation experiment, you could first measure the raw
muon stopping rate at an upper and lower elevation. Accounting for
energy loss between the two elevations, you first calculate the
transit time $t^{\prime}$ in the muon�s rest frame and then a
na\"{i}ve theoretical lower elevation stopping rate. This na\"{i}ve rate
should then be multiplied by the muon spectrum correction factor $1.5
\pm 0.2$ before comparing it to the measured rate at the lower
elevation. Alternatively, you could measure the lower elevation
stopping rate, divide by the correction factor, and then account for
energy loss before predicting what the upper elevation stopping rate
should be. You would then compare your prediction against a
measurement.

\section{Electronics}

\begin{figure}[tbp]
   \centering
   \includegraphics[width=5in]{MuonLifetime/Electronics}
   \caption{Block diagram of the readout electronics. The amplifier and 
   discriminator outputs are available on the front panel of the electronics 
   box. The HV supply is inside the detector tube. }
   \label{FigElectronics}
\end{figure}

A block diagram of the readout electronics is shown in figure
\ref{FigElectronics}. The logic of the signal processing is simple.
Scintillation light is detected by a photomultiplier tube (PMT) whose
output signal feeds a two-stage amplifier. The output from the
amplifier is then input into a voltage discriminator with an
adjustable threshold. If the input signal is above the discriminator
threshold, then the discriminator outputs a TTL (5 V) logic pulse that
triggers the timing circuit of the FPGA. If, within a fixed time
interval, the FPGA receives a second TTL pulse, then the timing
circuit will stop. The time interval between the start and stop timing
pulses is temprarily stored in a data buffer and then the timing
circuit is reset. (The reset takes about 1 msec during which the
detector is disabled.) The data stored in the buffer is sent to the PC
via the communications module. This data is used to determine the muon
lifetime. If a second TTL pulse does not arrive within the fixed time
interval, the timing circuit stores the value of this time interval in
the data buffer and then automatically resets the circuit for the next
measurement.


 
\begin{figure}[tbp]
   \centering
   \includegraphics[width=4.5in]{MuonLifetime/PhotoFrontBox}
   \caption{Front of the electronics box.}
   \label{FigPhotoFrontBox}
\end{figure}
\begin{figure}[tbp]
   \centering
   \includegraphics[width=4.5in]{MuonLifetime/PhotoRearBox}
\caption{Rear of electronics box. The communications ports are on the
left. Use only one.}
   \label{FigPhotoRearBox}
\end{figure}
\begin{figure}[tbp]
   \centering
   \includegraphics[width=3.5in]{MuonLifetime/PhotoDetectorTop}
\caption{Rear of electronics box. The communications ports are on the
left. Use only one.}
   \label{FigPhotoDetectorTop}
\end{figure}

The back panel of the electronics box is shown is figure
\ref{FigPhotoRearBox}. An extra fuse is stored inside the power
switch.

The front panel of the electronics box is shown in figure
\ref{FigPhotoFrontBox}. The amplifier output is accessible via the BNC
connector labeled \emph{Amplifier output}. Similarly, the
discriminator output is accessible via the connector labeled
\emph{Discriminator output}. The voltage level against which the
amplifier output is compared to determine whether the discriminator
triggers can be adjusted using the \emph{Threshold control} knob. The
threshold voltage is monitored by using the red and black connectors
that accept standard multimeter probe leads. The toggle switch
controls a beeper that sounds when an amplifier signal is above the
discriminator threshold. The beeper can be turned off.

Figure \ref{FigPhotoDetectorTop} shows the top of the detector
cylinder. DC power to the electronics inside the detector tube is
supplied from the electronics box through the connector \emph{DC Power}. The
high voltage (HV) to the PMT can be adjusted by turning the
potentiometer located at the top of the detector tube. The HV level
can be measured by using the pair of red and black connectors that
accept standard multimeter probes. The HV monitor output is 1/100
times the HV applied to the PMT.

A pulser inside the detector tube can drive a light emitting diode
(LED) imbedded in the scintillator. It is turned on by the toggle
switch at the tube top. The pulser produces pulse pairs at a fixed
repetition rate of 100 Hz while the time between the two pulses
comprising a pair is adjusted by the knob labeled \emph{Time Adj}. The pulser
output voltage is accessible at the connector labeled \emph{Pulse Output}.

\begin{figure}[tbp]
   \centering
   \includegraphics[width=4in]{MuonLifetime/PMTOutput}
\caption{Output pulse directly from PMT into a $50\, \Omega$ load.
Horizontal scale is 20 ns/div and vertical scale is 100 mV/div.}
   \label{FigPMTOutput}
\end{figure}

\begin{figure}[tbp]
   \centering
   \includegraphics[width=4in]{MuonLifetime/AmpOutput}
\caption{Amplifier output pulse from the input signal from figure
\ref{FigPMTOutput} and the resulting discriminator output pulse.
Horizontal scale is 20 ns/div and the vertical scale is 100 mV/div
(amplifier output) and 200 mV/div (discriminator output). }
   \label{FigAmpOutput}
\end{figure}

 
For reference, figure \ref{FigPMTOutput} shows the output directly
from the PMT into a $50\, \Omega$ load. Figure \ref{FigAmpOutput} shows
the corresponding amplifier and discriminator output pulses.




 \section{Data Acquisition}
 
To acquire the data, you will use the program \emph{Muon}. A link to
the program can be found on the desktop of the computer located on the
lab table. This program is rather basic. It collects the data,
validates it, and saves it to a file.  \emph{Muon} can provide an 
initial estimate of the time constant (muon lifetime) of the 
collected data.  However, to perform a proper calculation of the 
lifetime and its uncertainty, you will need to analyze the data in 
MatLab.  

\begin{figure}[tbp]
   \centering
   \includegraphics[width=4in]{MuonLifetime/MuonMainWindow}
\caption{Main window of the data acquisition program.}
   \label{FigMuonMainWindow}
\end{figure}

When you launch {\em muon} you will bring up the main window as shown 
in Figure \ref{FigMuonMainWindow}.  Within the window are five 
sections:
\begin{itemize}
   \item  Control

   \item  Muon Decay Time Histogram

   \item  Monitor

   \item  Rate Meter

   \item  Muons Through Detector
\end{itemize}

\begin{figure}[tbp]
   \centering
   \includegraphics[width=4in]{MuonLifetime/ConfigurePane}
   \caption{Muon's Configure Pane.}
   \label{FigConfigurePane}
\end{figure}

\subsection{Control}
Before data collection can begin, the system must be configured 
to know on which port the data will be received and how it should be 
histogrammed. Unfortunately these settings are not stored and must to 
reset with each run.  (At this point, you may better understand the 
description \emph{basic} that I applied to the program.)

By clicking the \emph{Configure} button, one opens the
\emph{Configure} pane as shown in Figure \ref{FigConfigurePane}. 
Select the port, which in our case, since we use USB, is com2.  The 
program does not properly store this data and sometimes resets the 
port to com1.  If, during the experiment, you are not receiving data, 
check that the correct port is selected.

Select the range and number of bins in the histogram.  During 
experimental set-up, good values to choose are 6 $\mu$s and 10 bins.  
However, for long data collection runs, choose 20 $\mu$s and 60 
bins.  The greater resolution provides a better check on the accuracy 
of your data.  Close the Configure pane,  the other sections of the 
pane are not relevant to data acquisition.

The \emph{Start} and \emph{Pause/Resume} buttons do exactly what 
their names imply. 

The \emph{Fit} button provides a crude fit to the data.  Use this 
only when the data acqusition is paused, since the collection of new 
data will erase the fit.  (N.B. The fit routine use in \emph{Muon} 
has had stability problems.  It's results are not sufficient for your 
lab report.)

The \emph{View Raw Data} button opens a window that allows you to
display the timing data for a user selected number of events, with the
most recent events read in first. Here an event is any signal above
the discriminator threshold so it includes data from both through
going muons as well as signals from muons that stop and decay inside
the detector. Each raw data record contains two fields of information.
The first is a time, indicating the year, month, day, hour, minute and
second, reading left to right, in which the data was recorded. The
second field is an integer that encodes two kinds of information. If
the integer is less than 40000, it is the time between two successive
flashes, in units of nanoseconds. If the integer is greater than or
equal to than 40000, then the units position indicates the number of
"time outs," (instances where a second scintillator flash did not
occur within the preset timing window opened by the first flash). See
the data file format below for more information. Typically, viewing
raw data is a diagnostic operation and is not needed for normal data
taking.

The \emph{Quit} button stops the measurement and asks you whether you
want to save the data. The questions seems rather straightforward, 
but the resulting action is as couterintuitive as pressing START to 
shutdown a computer.  Answering \emph{NO} writes the data to a file that is
named after the date and time the measurement was originally started,
i.e., 03-07-13-17-26.data. Answering \emph{YES} appends the data to the file
muon.data. The file muon.data is intended as the main data file.  I 
recommend performing a test run to see what happens in both cases.  
As I prefer having the datafile name include the data, I often answer 
\emph{NO}.  Others prefer choosing \emph{YES}, then renaming the 
filename.  

\subsubsection{Data File Format}

Timing information about each signal above threshold is written to 
the data file.  The file is in ascii format and can be read by any 
text editor.  The first data field provides information on the time 
between sucessive pulses. If that integer is less than 40000, then 
the value corresponds to the time, to the nearest nanosecond, between 
successive signals.  If the value is greater than or equal to 40000
then the timing circuit exceeded its maximum number of clock cycles 
before receiving a second signal.  The circuit \emph{timed-out}.  The 
number in the units place indicates the number of times the circuit 
timed-out before receiving a signal.  For example, 40005 corresponds 
to a circuit that was triggered but then timed-out 5 times before a 
subsequent signal arrived.  

The second field is the computer time in seconds.  This value is 
typically the number of seconds since  1 January 1970, a 
date convention used in C and in Unix. 


\subsection{Monitor} 
This panel shows rate-related information for the current measurement.
The elapsed time of the current measurement is shown along with the
accumulated number of times from the start of the measurement that the
readout electronics was triggered (Number of muons). The Muon Rate is
the number of times the readout electronics was triggered in the
previous second. The number of pairs of successive signals, where the
time interval between successive signals is less than the maximum
number of clock cycles of the timing circuit, is labeled Muon Decays,
even though some of these events may be background events and not real
muon decays. Finally, the number of muon decays per minute is
displayed as Decay Rate.

\subsection{Rate Meter} 
This continuously updated graph plots the number of signals above
discriminator threshold versus time. It is useful for monitoring the
overall trigger rate.

\subsection{Muons through Detector}
This graph shows the time history of the number of signals above
threshold. Its time scale is automatically adjusted and is intended to
show time scales much longer than the rate meter. This graph is useful
for long term monitoring of the trigger rate. Strictly speaking, it
includes signals from not only through going muons but any source that
might produce a trigger. The horizontal axis is time, indicated down
to the second. The scale is sliding so that the far left-hand side
always corresponds to the start of the measurement session. The bin
width is indicated in the upper left-hand portion of the plot.

\subsection{Muon Decay Time Histogram}
This plot is probably the most interesting one. It is a histogram of
the time difference between successive triggers and is the plot used
to measure the muon lifetime. The horizontal scale is the time
difference between successive triggers in units of microseconds. Its
maximum displayed value is set by the \emph{Configure} menu. (All time
differences less than 20 $\mu$sec are entered into the histogram but
may not actually be displayed due to menu choices.) You can also set
the number of horizontal bins using the same menu. The vertical scale
is the number of times this time difference occurred and is adjusted
automatically as data is accumulated. A button (\emph{Change y scale
Linear/Log}) allows you to plot the data in either a linear-linear or
log-linear fashion. The horizontal error bars for the data points span
the width of each timing bin and the vertical error bars are the
square root of the number of entries for each bin. The upper right
hand portion of the plot shows the number of data points in the
histogram. Again, due to menu selections not all points may be
displayed. If you have

selected the \emph{Fit} button then information about the fit to the
data is displayed. The muon lifetime is returned, assuming muon decay
times are exponentially distributed, along with the chi-squared per
degree of freedom ratio, a standard measure of the quality of the fit.
(See Bevington for more details.) 

A \emph{Screen capture} button allows you to produce a plot of the display.
Select the button and then open the \emph{Paint} utility (in Windows) and
paste the graphic into a blank document window.


\section{Procedure}

\subsection{Getting Started}

The black aluminum cylinder (�detector�) is placed in the wooden pedestal for 
convenience; it will work in any orientation. 
Connect the power cable and signal cable between the electronics box and the detector. 
Connect the USB cable between the back of the electronics box and the 
PC.  If USB is not available then use the serial cable. \textbf{DO NOT USE 
BOTH.}

Turn on power to the electronics box. The red LED indicates that the 
power is on.  The green LED indicates an event in the PMT.  

Set the HV between  �1100 and �1200 Volts using the knob at the top of the detector 
tube. The exact setting is not critical and the voltage can be monitored by using the 
multimeter probe connectors at the top of the detector tube. 

If you are curious, you can look directly at the output of the PMT
using the \emph{PMT Output} on the detector tube and an oscilloscope.
(A digital scope works best.) \textbf{Be certain to terminate the
scope input at 50 }$\mathbf{\Omega}$\textbf{ or you signal will be
distorted}. You should see a signal that looks like Figure \ref{FigPMTOutput}. The
figure shows details like scope settings and trigger levels.


Connect the BNC cable between \emph{PMT Output} on the detector and
\emph{PMT Input} on the box. Adjust the discriminator setting on the
electronics box so that it is in the range 180--220 mV. The green LED
on the box front panel should now be flashing.

You can look at the amplifier output by using the \emph{Amplifier
Output} on the box front panel and an oscilloscope. The scope input
impedance must be 50 $\Omega$. Similarly, you can examine the output
of the discriminator using the \emph{Discriminator Output} connector.
Again, the scope needs to be terminated at 50 $\Omega$. Figure
\ref{FigAmpOutput} shows typical signals for both the amplifier and
discriminator outputs on the same plot. Details about scope time
settings and trigger thresholds are on the plot.


Launch the program \emph{Muon} which is located on the desktop of the 
PC.  Configure the program using the \emph{Configure} button to set 
the data port and the histogram dimensions.

Click on \emph{Start}. You should see the rate meter at the lower left-hand side of your computer 
screen immediately start to display the raw trigger rate for events that trigger the readout 
electronics. The mean rate should be about 6 Hz or so.

\subsection{Lab Exercises}

\begin{enumerate}
   \item  Measure the gain of the 2-stage amplifier using a sine wave. 
    Apply a 100kHz 100mV peak-to-peak sine wave to the input of the electronics box 
input. Measure the amplifier output and take the ratio Vout/Vin. 
    
Increase the frequency. How good is the frequency response of the amp? 

Estimate the maximum decay rate you could observe with the instrument.

   \item  Measure the saturation output voltage of the amp. 
   Increase the magnitude of the input sine wave and monitor the amplifier output. 
  Does a saturated amp output change the timing of the FPGA? What are the 
implications for the size of the light signals from the scintillator? 

   \item  Examine the behavior of the discriminator by feeding a sine wave to the box input and 
adjusting the discriminator threshold. Monitor the discriminator output and describe its 
shape.

   \item  Measure the timing properties of the FPGA: 
   \begin{enumerate}
\item Using the pulser on the detector, measure the time between
successive rising edges on an oscilloscope. Compare this number with
the number from software display.
   
      \item  Measure the linearity of the FPGA: 
      Alter the time between rising edges and plot scope results v. FPGA results; 
      Can use time between 1 $\mu$s and 20 $\mu$s in steps of 2 $\mu$s. 
   
      \item   Determine the timeout interval of the FPGA by gradually increasing the time between 
successive rising edges of a double-pulse and determine when the FPGA no longer 
records results. What does this imply about the maximum time between signal pulses? 
   
      \item  Decrease the time interval between successive pulses and try to determine/bound the 
FPGA internal timing bin width. 
  What does this imply about the binning of the data? 
What does this imply about the minimum decay time you can observe?
   \end{enumerate}
 

   \item  Adjust (or misadjust) discriminator threshold. 
Increase the discriminator output rate as measured by the scope or some other means. 
Observe the raw muon count rate and the spectrum of "decay" times. (This exercise needs 
a digital scope and some patience since the counting rate is �slowish.�)

   \item  What HV should you run at? Adjust/misadjust HV and observe amp output. (We know 
that good signals need to be at about 200 mV or so before discriminator, so set 
discriminator before hand.) With fixed threshold, alter the HV and watch raw muon count 
rate and decay spectrum. 


\item Connect the output of the detector can to the input of the
electronics box. Look at the amplifier output using a scope. (A
digital scope works best.) Be sure that the scope input is terminated
at 50 $\Omega$. What do you see? Now examine the discriminator output
simultaneously. Again, be certain to terminate the scope input at 50 $\Omega$.
What do you see?

   \item   Set up the instrument for a muon lifetime measurement. 
   Start and observe the decay time spectrum. 
   The muons whose decays we observe are born outside the detector and therefore 
spend some (unknown) portion of their lifetime outside the detector. So, we never 
measure the actual lifetime of any muon. Yet, we claim we are measuring the lifetime of 
muons. How can this be?

   \item Write a MatLab routine to fit the decay time histogram and 
   the background.

   \item   From your measurement of the muon lifetime and a value of the muon mass from 
some trusted source, calculate the value of Fermi coupling constant 
$G_{F}$. Compare your 
value with that from a trusted source.

   \item Using the approach outlined in the text, measure the charge
   ratio $\rho$ of positive to negative muons at ground level or at
   some other altitude.
   
\end{enumerate}

\endinput
