
The Michelson Interferometer

History of the Michelson Interferometer

Theory of Michelson Interferometer

Experiment with Michelson Interferometer

Data from Experiment

Other Uses for the Michelson Interferometer

History:

   Back in the early days, the only waves that people could observe were those travelling in some sort of medium.  For example, the waves in a river or the ocean were seen travelling in water.  Therefore, there natural conclusion to be made by many people (Most importantly Huygen) that light, if  it were to be a wave, and it had been proven to exhibit characteristics of them, would have to be travelling in some medium.  As history would dictate, this mysterious medium that nobody could perceive with their senses or instrument, would be deemed the "luminiferous ether".  Light, as well as all of the solar systems in the universe were said to be ensconced in this ether which would have to maintain a wave speed of 3e8 m/s (according to Maxwell's predictions at the time).   The team of the two professors,   Michelson and Morley, would attempt to detect the presence of the elusive ether in 1880.  To do so, they devised the interferometer which would allow them to observe the effect of the motion of the medium upon the propagation of light.   Hence, interferometry was born around the famous "null outcome" experiment performed by Michelson and Morley in the late nineteenth century.  

Back to Top of Page...

Theory of the Michelson Interferometer:

    The picture below represents an interferometer in its simplest form:

        

A source S emits monochromatic light that hits the surface M at point C.  M is partially reflective, so one beam is transmitted through to point B while one is reflected in the direction of A.  Both beams recombine at point C to produce an interference pattern (assuming proper alignment) visible to the observer at point E.  To the observer at point E, the effects observed would be the same as those produced by placing surfaces A and B' (the image of B on the surface M) on top of each other.  Let's look at this interaction in more detail.  Imagine that we two surfaces M1 and M2 as diagramed below:



    The light of interest originates at point S until it comes into contact with the two surfaces (we assume the angles of incidence are the same) and is reflected to point P.  Here theta is the angle of incidence for the light, t represents the separation of the two surfaces, and phi is the angle between them which can be assumed small enough to ignore.  The difference in paths of the two beams, delta, originating at S can then be described by .  If we make the intersection of the planes M1 and M2 be vertical, the distance between the mirrors to P is d, and the angle between the normal to M1 i, then the above expression can be written as   (assuming i is small).  Now one can write an expression for the phase difference for the light, alpha,  .   This equation, however, presents quite a wide range of values for the phase which might make it impossible, or at least very hard to observe interference phenomena because we want the phase difference to be small (see my explanation why in the discourse on light basics).  Therefore,  delta must be for a particular change in the angle of incidence, theta.  This modification can be done by taking the derivative of delta with respect to theta and setting it equal to zero.   In doing so, we find that the different parts of the interference pattern can't be seen unless the distance between the two planes M1 and M2, t0 ,is zero when d=0 (case where interference effects are localized at M1 and M2),  or if the angle between the two plane surfaces is zero and P=infinity.  Intuitively, this makes sense because the smaller the distance between the two planes, the smaller the distance for the phases of the two rays to get less uniform.  From this situation, it also becomes possible to determine the shape of the interference effects.  As it turns out, the effects are either linear or circular patterns.  Lets redraw the coordinate above to see mathematically how to obtain these fringe patterns (i.e. the situation pictured below).

 

  One now looks at the two planes from the front instead of the side.  The observer resides at point O, R is the base of the perpendicular from O onto the plane M1, and the ray OR=z.  The path difference, as we have proved above, from M1 to M2 is .  The ray OP is the hypotenuse of the triangle OPR so it is equal to .  The cosine of the angle theta, which will be important in a moment, can then be expressed as .  The surface thickness at point R will be designated "e", so at the point P(x,y)   the thickness is defined to be , where alpha is the angle that separates the two planes (It was phi in the diagram above it).  For stationary waves, the normal modes occur at integer multiples of the lights' wavelength given by the following formula .  If we had a wave on a string, L would the length of the string itself.   In our case, however, it is the distance from the plane M2 to the plane M1.   Thus, this distance .  Now we can write an expression that governs the behavior of the fringes that we see at point O.  This relationship is

 . 

    Consider the two important cases that determine the shape of the interference patterns observed at point O.  The first case occurs when both M1 and M2 intersect at the perpendicular, point R.  Here the thickness of the surfaces, e, is zero.  Thus,  the  fringe determining equation is rearranged to .  The relationship was able to be reduced by making the assumption that z >>x+y and alpha is small. The resulting expression is that of a line with fringes separated by distance .   Interestingly enough, if the observer is too close to the surface planes M1 and M2, this assumption can't be made.  The result is a hyperbolic function causing the fringes to appear slightly bent. 

    The second case to be considered appears when the two surface planes are parallel to each other (i.e. the angle alpha between them is zero).   Thus,our fringe governing equation gets rearranged yet another time to yield:

 

Clearly this represents the equation of a circle, hence our fringes appear circular to the observer.

    One more important point to remember is the property of coherence length when it comes to a device that studies interference effects.  If the differences in path length to both the mirrors and back from the beamsplitter  is greater than the coherence length, then no interference effects will be seen.  Thus, when adjusting the moveable mirror on the Michelson interferometer, one must be precise when trying to find interference patterns of light comprised of a large difference in wavelengths, such as white light. Now that you understand how we get these fringes, lets see if we can reproduce them with a Michelson Interferometer!

Back to Top of Page..

Studying the Wave Nature of Light with a Michelson Interferometer:

    The Michelson Interferometer represents a device that takes advantage of the Wave Nature of Light.  If light were not to be considered a wave, none of the observed interference patterns could occur in experiments as they do.  In this section,  how light can interact and interfere with itself to produce these fringes and patterns characteristic those of waves is described.  Below (thanks to my partner on the project that we performed this semester, Seth Carpenter) we see an interferometer a little more complex than that used in the theory discussion.



The main difference between the one used to derive theory and the one picture above, is the addition of the "compensatory lens".   This lens is added so that both paths, from the back of beam splitter to Mirror 1 and Mirror 2, are identical.  This addition becomes necessary since the back of the beam splitter is where the light gets reflected (usually by a partially reflective silver coating).   Thus, the light travelling to Mirror 1 travels through the thickness of the beamsplitter 3 times; once as it enters initially, next as it is reflected off the back of it, and finally as it returns to the beamsplitter after reflecting off of Mirror 1.   If the compensatory lens were not in place, the light travelling to Mirror 2 would only travel through the beamsplitter once since it is transmitted through it and then reflected off the back of it.  Therefore, the compensatory lens MUST be of exactly the same thickness as the beam splitter.  Below is an actual picture of the Michelson Interferometer that we used to accumulate our data:



 

    The light originates from some source and is incident from the left.   The interference pattern can be observed where the white piece of paper is at the bottom of the picture.  Continue below to observe actual fringe patterns acquired using this apparatus. 

Back to Top of Page...

Some Data From Previous Experiments:

    Here are some pictures of fringes, both hyperbolic and circular in nature, as predicted by the theory discussed earlier.





    The top pictures are the interference patterns of a HeNe laser, the bottom are of sulfur.  Clearly, they coincide with what theory predicts.  Using the HeNe laser as a reference, one can calibrate the Michelson interferometer so that it can be used to determine the unknown wavelength(s) of the Sodium light. This is possible because of orders of interference.  From general wave theory, constructive interference occurs in half multiples of the wavelength.  Using this relationship, , we can solve for the wavelength of light because this ratio represents the variable L.  Thus, to find the wavelength, let a motor drive the lever arm while a device is used to count the fringes that pass by (the variable n in the equation).  Here are the results that myself and Dr. Von Seth Carpenter obtained while attempting this exercise:







    Stepping through the above data, we first determined how far the mirror traveled for a specific number of HeNe fringes to pass by.  Since we knew the wavelength of a HeNe laser very precisely to be 632.8 nm, we used this relationship defined above solving for L.  Next, we calculated the ratio of how far the lever arm moves in mm to how far the actual interferometer mirror gets displaced.  This value allows one determine the unknown wavelength mostly because one determined the length L in terms of how far the mirror moves by looking at how far the lever arm turns.   For sodium, we observed 210 fringes pass as the lever arm moved .315 mm.   Again using our relationship for orders of constructive interference, we determine the unknown wavelength of sodium to be on the order of 606.905 nm.  In looking at the American Institute of Physics Handbook, the most probable wavelength of light we should see is 589.5 nm with a 62.8% probability.  The other line that we should have seen, but didn't was 588.9 nm with 63% transition probability.  Our 606.905 nm calculated data is within  -2.96% error.  Finally, a phenomena that occurs with sodium, but not the HeNe light (because it is close one wavelength), are beats.  Beats occur when two waves of different frequencies are travelling in the same direction and interact with each other.  This means that they alternate between constructive and destructive interference because they are periodically out of phase.  Since the sodium light is comprised of the two different wavelengths, 589.95 and 588.5 nm, it finds itself susceptible to this temporal interference. 

Back to Top of Page...

Other Uses for the Michelson Interferometer:

This link reveals a new method for measuring gravity using convection currents in fluids.  They use the Wave theory of light as it pertains to light passing through these convection currents with different indices of refraction...pretty cool stuff.   NASA sponsored it so you know its ripe.  Check it out here.

 

Here, the Michelson Interferometer is used to measure trace gases in our atmosphere by Passive Atmospheric Sounding.  You can see it here.

 

How big can an interferometer get?  A 20 feet? 100 feet?  Well, they have them both here.  They are used to measure the diameters of various stars and planets.   Read all about it here.

Back to Top of Page...

References:

Gray, Dwight E. ed.  American Institute of Physics Handbook 3rd Ed.  McGraw-Hill: New York. 1982. 
Michelson, A. A.  Studies In Optics.   University of Chicago Press:  Chicago.  1927.
Serway, Raymond A.  Physics For Scientists and Engineers with Modern Physics.  Saunders College Publishing:   Philadelphia.  1996. 
Tolansky, S.  An Introduction to Interferometry. William Clowes and Sons Ltd.:  London.  1966.
Table of Contents:

Abstract
Travelling Michelson Interferometer
FraunhoferDiffraction
Fabry-Perot Interferometer
Cabell's Main Web Page...